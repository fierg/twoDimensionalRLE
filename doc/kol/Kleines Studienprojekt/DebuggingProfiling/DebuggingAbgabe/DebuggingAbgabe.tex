\documentclass[10pt, a4paper]{article}
\usepackage[utf8]{inputenc}
\usepackage{amsrefs}
\usepackage{tikz, wasysym}
\usetikzlibrary{automata,positioning}
\usepackage[]{algorithm2e}
\usepackage[]{csquotes}
\usepackage{numberedblock}

\title{Debugging}
\date{\today}

%TODOs: fix empfy last page
%		fill placeholher

\begin{document}
\begin{center}
\section*{Debugging}
\end{center}
\textbf{Sven Fiergolla 1252732}
\section*{Aufgabe a)}
\paragraph{}
Erstes Starten der Anwendung wirft sofort \enquote{\textit{java.lang.ArithmeticException: / by zero}}. Betrachten der Methode \textit{computePi()} $\rightarrow$\\
Anpassen der Zeile 16 zu: \\
\begin{numVblock}
piDividedBy4 += sign * (1 / (i + 1));
\end{numVblock}
entsprechend der Leibniz-Reihe zur Berechnung.\par
\bigskip

\paragraph{Debugging}
Nun ist normales Debugging möglich. Aufruf mit Argument = 5 um erste Werte mit tatsächlichem Ergebnis zu vergleichen.\\
k = 0, result = 1 $\checked$\\
k = 1, result = 1 $\times$\\
Breakpoint zu Beginn der Methode, Fehler in Zeile\\
\begin{numVblock}
piDividedBy4 += sign * (1 / (i + 1));
\end{numVblock}
da $i$ vom Typ int ist, wird in der 2. Iteration wo $i = 2$ ist, aus $\frac{1}{i + 1}$ = $\frac{1}{2 + 1}$ statt $\frac{1}{3}$ jedoch 0, da int keine Brüche fassen kann.\\
$\rightarrow$ verändere i zum Typ double!\\
\begin{numVblock}
for (double i = 0; i <= iteration * 2; i += 2) {
piDividedBy4 += sign * (1 / (i + 1));
sign *= -1;
}
\end{numVblock}\paragraph{•}
\paragraph{Weitere Anpassungen}
Methodenname = computePi(), Funktion gibt jedoch $\frac{\pi}{4}$ zurück!\\
\begin{numVblock}
return piDividedBy4 * 4;
\end{numVblock}
\par
\paragraph{}
Eingabe von negativen Zahlen führt zu ungewünschtem Ergebnis, negative Interationen nicht möglich!\\
Abfrage auf iteration \textgreater 0, sonst:\\
\begin{numVblock}
} else {
throw new IllegalArgumentException()
}
\end{numVblock}\par
\end{document}

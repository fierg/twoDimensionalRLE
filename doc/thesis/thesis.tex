\documentclass{wissdoc}
% Autor: Roland Bless 1996-2009, bless <at> kit.edu
% ----------------------------------------------------------------
% Diplomarbeit - Hauptdokument
% ----------------------------------------------------------------
%%
%% $Id: thesis.tex 65 2012-05-10 10:32:11Z bless $
%%
%
% Zum Erstellen zweiseitiger PDFs (für Buchdruck) in der Datei "wissdoc.cls" folgende Zeile abändern:
%
% \LoadClass[a4paper,12pt,oneside]{book} % diese Klasse basiert auf ``book''
% in
%\LoadClass[a4paper,12pt,titlepage]{book} % diese Klasse basiert auf ``book''
%
%
% wissdoc Optionen: draft, relaxed, pdf --> siehe wissdoc.cls
% ------------------------------------------------------------------
% Weitere packages: (Dokumentation dazu durch "latex <package>.dtx")
\usepackage[numbers,sort&compress]{natbib}
\usepackage{hyperref}
\usepackage{amsfonts}
\usepackage{csquotes}
\usepackage[english]{babel}
\usepackage{tikz}
\usetikzlibrary{arrows}
\usetikzlibrary{arrows.meta}
\usepackage{wrapfig}
\usepackage{pgfplots}
\usepackage{readarray}
\usepackage{changes}
\usepackage{amsmath}
\usepackage{mathtools}
\mathtoolsset{showonlyrefs}  


\usepackage{float}
\restylefloat{table}

\usepackage{environ}
\makeatletter
\newsavebox{\measure@tikzpicture}
\NewEnviron{scaletikzpicturetowidth}[1]{%
	\def\tikz@width{#1}%
	\def\tikzscale{1}\begin{lrbox}{\measure@tikzpicture}%
		\BODY
	\end{lrbox}%
	\pgfmathparse{#1/\wd\measure@tikzpicture}%
	\edef\tikzscale{\pgfmathresult}%
	\BODY
}
\makeatother




\def\data{% the data
0 1 1 0 0 0 0 1 
0 1 1 0 0 0 1 0 
0 1 1 1 0 0 1 0 
0 1 1 0 0 0 0 1 
0 1 1 0 0 0 1 1 
0 1 1 0 0 0 0 1 
}
\readarray\data\dataA[6,8] %read the data to \dataA

\def\data{% the data
0 1 0 0 1 1 0 0 
0 1 1 0 1 1 1 1 
0 1 1 1 0 0 1 0 
0 1 1 0 0 1 0 1 
0 1 1 0 1 1 0 1 
0 0 1 0 0 0 0 0 
0 1 1 0 1 0 0 1 
0 1 1 1 0 0 0 0 
0 1 1 1 0 0 1 1 
0 1 1 1 0 1 0 1 
0 1 1 0 1 1 0 1 
0 0 1 0 0 0 0 0 
0 1 1 0 0 1 0 0 
0 1 1 0 1 1 1 1 
0 1 1 0 1 1 0 0 
0 1 1 0 1 1 1 1 
0 1 1 1 0 0 1 0 
0 0 1 0 0 0 0 0 
0 1 1 1 0 0 1 1 
0 1 1 0 1 0 0 1 
0 1 1 1 0 1 0 0 
0 0 1 0 0 0 0 0 
0 1 1 0 0 0 0 1 
0 1 1 0 1 1 0 1 
0 1 1 0 0 1 0 1 
0 1 1 1 0 1 0 0 
0 0 1 0 1 1 1 0  
}
\readarray\data\dataB[26,8]

\def\data{% the data
0 0 0 0 1 0 0 1 
0 0 0 0 0 0 1 0 
0 0 0 0 0 1 0 1 
0 0 0 0 0 0 1 1 
0 0 0 0 0 0 0 1 
0 0 0 0 0 0 0 0 
0 0 0 0 0 1 0 0 
0 0 0 0 1 1 0 1 
0 0 0 0 0 1 1 0 
0 0 0 0 1 1 1 0 
0 0 0 0 0 0 0 1 
0 0 0 0 0 0 0 0 
0 0 0 0 1 0 1 1 
0 0 0 0 0 0 1 0 
0 0 0 0 1 1 0 0 
0 0 0 0 0 0 1 0 
0 0 0 0 0 1 0 1 
0 0 0 0 0 0 0 0 
0 0 0 0 0 1 1 0 
0 0 0 0 0 1 0 0 
0 0 0 0 0 1 1 1 
0 0 0 0 0 0 0 0 
0 0 0 0 1 0 1 0 
0 0 0 0 0 0 0 1 
0 0 0 0 0 0 1 1 
0 0 0 0 0 1 1 1 
0 0 0 0 1 0 0 0 
}
\readarray\data\dataC[26,8]

\tikzset{
	treenode/.style = {align=center, inner sep=0pt, text centered,
		font=\sffamily},
	arn_n/.style = {treenode, circle, white, font=\sffamily\bfseries, draw=black,
		fill=black, text width=1.5em},% arbre rouge noir, noeud noir
}

% \usepackage{varioref}
% \usepackage{verbatim}
% \usepackage{float}    %z.B. \floatstyle{ruled}\restylefloat{figure}
% \usepackage{subfigure}
% \usepackage{fancybox} % für schattierte,ovale Boxen etc.
% \usepackage{tabularx} % automatische Spaltenbreite
% \usepackage{supertab} % mehrseitige Tabellen
% \usepackage[svnon,svnfoot]{svnver} % SVN Versionsinformation 
%% ---------------- end of usepackages -------------

%\svnversion{$Id: thesis.tex 65 2012-05-10 10:32:11Z bless $} % In case that you want to include version information in the footer

%% Informationen für die PDF-Datei
\hypersetup{
 pdfauthor={Sven Fiergolla},
 pdftitle={Improvements on RLE by preprocessing}
 pdfsubject={Not set},
 pdfkeywords={Not set}
}

% Macros, nicht unbedingt notwendig
\input{macros}

% Print URLs not in Typewriter Font
\def\UrlFont{\rm}

\newcommand{\blankpage}{% Leerseite ohne Seitennummer, nächste Seite rechts
 \clearpage{\pagestyle{empty}\cleardoublepage}
}

%% Einstellungen für das gesamte Dokument

% Trennhilfen
% Wichtig! 
% Im ngerman-paket sind zusätzlich folgende Trennhinweise enthalten:
% "- = zusätzliche Trennstelle
% "| = Vermeidung von Ligaturen und mögliche Trennung (bsp: Schaf"|fell)
% "~ = Bindestrich an dem keine Trennung erlaubt ist (bsp: bergauf und "~ab)
% "= = Bindestrich bei dem Worte vor und dahinter getrennt werden dürfen
% "" = Trennstelle ohne Erzeugung eines Trennstrichs (bsp: und/""oder)

% Trennhinweise fuer Woerter hier beschreiben
\hyphenation{
% Pro-to-koll-in-stan-zen
}

% Index-Datei öffnen
\ifnotdraft{\makeindex}

\begin{document}

\frontmatter
\pagenumbering{roman}
\ifnotdraft{
 %% Titelseite
%% Vorlage $Id: titelseite.tex 61 2012-05-03 13:58:03Z bless $

\def\usesf{}
\let\usesf\sffamily % diese Zeile auskommentieren für normalen TeX Font

\newsavebox{\Erstgutachter}
\savebox{\Erstgutachter}{\usesf Prof.~Dr.~Ingo~J.~Timm}
\newsavebox{\Zweitgutachter}
\savebox{\Zweitgutachter}{\usesf xxxxxxxxxx}
\newsavebox{\Betreuer}
\savebox{\Betreuer}{\usesf xxxxxxxxxx}

\begin{titlepage}
\setlength{\unitlength}{1pt}
\begin{picture}(0,0)(85,770)
%\includegraphics[width=\paperwidth]{logos/KIT_Deckblatt}
\end{picture}

\thispagestyle{empty}

%\begin{titlepage}
%%\let\footnotesize\small \let\footnoterule\relax
\begin{center}
\hbox{}
\vfill
{\usesf
{\huge\bfseries String Run Length Encoding (RLE) on bit level against regular RLE\par}
\vskip 1.8cm
{\huge Bachelorarbeit}\\
\vskip 0.5cm
zur Erlangung des akademischen Grades\\
Bachelor of Science (B.Sc.) 
\vskip 1.5cm

{\large Universität Trier\\
FB IV - Informatikwissenschaften\\
Lehrstuhl für Informatik I\\}

\vskip 3cm
\begin{tabular}{p{3.5cm}l}
Gutachter: & \usebox{\Erstgutachter} \\
 & \usebox{\Zweitgutachter} \\
Betreuer: & \usebox{\Betreuer} \\
\end{tabular}
\vskip 3cm
Vorgelegt am xx.xx.xxxx von:\\
\vskip .5cm
Sven Fiergolla\\
Am Deimelberg 30\\
54295 Trier\\
sven.fiergolla@gmail.com\\
Matr.-Nr. 1252732


}
\end{center}
\vfill
\end{titlepage}
%% Titelseite Ende


%%% Local Variables: 
%%% mode: latex
%%% TeX-master: "thesis"
%%% End: 

 \blankpage % Leerseite auf Titelrückseite
 \chapter*{Zusammenfassung}
%% ==============================
Hier steht eine Kurzzusammenfassung (Abstract) der Arbeit. Stellen Sie kurz und präzise Ziel und Gegenstand der Arbeit, die angewendeten Methoden, sowie die Ergebnisse der Arbeit dar. Halten Sie dabei die ersten Punkten eher kurz und fokussieren Sie die Ergebnisse. Bewerten Sie auch die Ergebnissen und ordnen Sie diese in den Kontext ein.

Die Kurzzusammenfassung sollte maximal 1 Seite lang sein.

In this paper, we consider a commonly used compression scheme
called run-length encoding. We provide both lower and upper
bounds for the problems of comparing two run-length encoded
strings. Specifically, we prove the 3sum-hardness for both the
wildcard matching problem and the k-mismatch problem with
run-length compressed inputs. Given two run-length encoded
strings of m and n runs, such a result implies that it is very
unlikely to devise an o ( mn ) -time algorithm for either of them. We
then present an inplace algorithm running in O ( mn log m ) time
for their combined problem, i.e. k-mismatch with wildcards. We
further demonstrate that if the aim is to report the positions of
all the occurrences, there exists a stronger barrier of Ω ( mn log m ) -
time, matching the running time of our algorithm. Moreover, our
algorithm can be easily generalized to a two-dimensional setting
without impairing the time and space complexity.
}
%
%% *************** Hier geht's ab ****************
%% ++++++++++++++++++++++++++++++++++++++++++
%% Verzeichnisse
%% ++++++++++++++++++++++++++++++++++++++++++
\ifnotdraft{
{\parskip 0pt\tableofcontents} % toc bitte einzeilig
%\blankpage

\listoffigures
%\blankpage
\listoftables
%\blankpage
}


%% ++++++++++++++++++++++++++++++++++++++++++
%% Hauptteil
%% ++++++++++++++++++++++++++++++++++++++++++
\graphicspath{{img/}}

\mainmatter
\pagenumbering{arabic}
%% Einleitung.tex
%% $Id: einleitung.tex 61 2012-05-03 13:58:03Z bless $
%%

\chapter{Introduction}
\label{ch:Introduction}
%% ==============================
Die Einleitung besteht aus der Motivation, der Problemstellung, der Zielsetzung und einem erster Überblick über den Aufbau der Arbeit. \\
TODO: \\
- explain compression ratio \\
- define unit of compression to quantify question and results \\

%% ==============================
\section{Motivation}
%% ==============================
\label{ch:Introduction:sec:Motivation}

In the last decades, digital data transfer became available everywhere and to everyone. This rise of digital data urges the need for data compression techniques or improvements on existing ones. Run-length encoding \cite{rle-patent} (abbreviated as RLE) is simple coding schemes that performs lossless data compression. RLE compression simply represents the consecutive, identical symbols of
a string with a run, usually denoted by $\sigma \ i$, where $\sigma$ is an alphabet symbol and $i$ is its number of repetitions. To give an example, the string aaaabbaaabbbba can be compressed into RLE format as  $ a^{4}b^{2}a^{3}b^{4}a^{1}$ . Its simplicity and efficiency make run-length encoding is still used in several areas like fax transmission, where RLE compression is combined with other techniques into Modified Huffman Coding \cite{fax-rle}. Most fax documents are typically simple texts on a white background, RLE compression is particularly suitable for fax and often achieves good compression ratios. Another appliance of RLE is optical character recognition, in which the inputs are usually images of large scales of identically valued pixels. Other applications appear in bioinformatics, where RLE compression is employed to speed up the comparison of two biological sequences.

%% ==============================
\section{Problem statement}
%% ==============================
\label{ch:Introduction:sec:Problem statement}

Some strings like aaaabbbb archive a very good compression rate because the string only has two different characters and they repeat at least twice. Therefore it can be compressed to $a^4b^4$ so from 8 byte down to 4 bytes if you encode it properly. On the other hand, if the input is highly mixed characters with few or no repetitions at all like abababab, the run length encoding of the string is $a^1b^1a^1b^1a^1b^1a^1b^1a^1b^1$ which needs at least 16 bytes. \par
So the inherent problem with run length encoding is obviously the possible explosion in size, due to missing repetitions in the input string. Expanding the string to twice the original size is not really a good compression so one has to make sure the input data is fitted for RLE as compression scheme. One goal is to minimize the increase in size in the worst case scenario. \par
Also it should improve the compression ratio on data suited for run length encoding and perform better than the originally proposed RLE.

%% ==============================
\section{Main Objective}
%% ==============================
\label{ch:Introduction:sec:Main Objective}

Was ist das Ziel der Arbeit. Wie soll das Problem gelöst werden?

- bessere kompressionsrate im vergleich zu konventionellem RLE\\
- gleiche oder ähnliche decoding zeit ?\\
- ansatz beschreiben\\

The main objectives that derives from the problem statement is to archive an improved compression ratio compared to regular run length encoding. To unify the measurements, the compression ratio is calculated by encoding all files listed in the Galgary Corpus and then normalize the results. \par
Since most improvements like permutations on the input, for example a revertable Burros-Wheeler transformation to increase the number of consecutive Symbols or a different way of reading the bytestream take quite some time, encoding speed will increase. A secod objective might be to keep decoding speed close to the original run legth encoding.

%% ==============================
\section{Structure of this work}
%% ==============================
\label{ch:Intoduction:sec:Structure}

Was enthalten die weiteren Kapitel? Wie ist die Arbeit aufgebaut? Welche Methodik wird verfolgt?

- describe following structure\\
- use references\\
- try to keep idea of a recurrent theme\\

%%% Local Variables: 
%%% mode: latex
%%% TeX-master: "thesis"
%%% End: 
  % Einleitung
%% grundlagen.tex
%% $Id: grundlagen.tex 61 2012-05-03 13:58:03Z bless $
%%

\chapter{Basic principles of compression}
\label{ch:Basic principles of compression}
%% ==============================

To understand compression one first hast to understand some basic pricipels of information theory like Entropy and different approaches to compress different types of data with different encoding and entropy. I will also show the key differences between probability coding and dictionary coding and a few comments on lossy compression.

%% ==============================

\section{Compression and Encoding fundamentals}
%% ==============================
\label{ch:Basic principles of compression:sec:Compression}

...
\subsection{Entropy and Unit of Compression}
\subsection{Probability Coding}
\subsection{Dictionary coding}
\subsection{Irreversible Compression}


%% ==============================
\section{Run Length Encoding}
%% ==============================
\label{ch:Grundlagen:sec:Run Length Encoding}

...

%% ==============================
\section{Huffman Coding}
%% ==============================
\label{ch:Grundlagen:sec:Huffman Coding}

...

%% ==============================
\section{State of the Art}
%% ==============================
\label{ch:Grundlagen:sec:SOTA}
Die Literaturrecherche soll so vollständig wie möglich sein und bereits existierende relevante Ansätze (Verwandte Arbeiten / State of the Art / Stand der Technik) beschreiben bzw. kurz vorstellen.
Es soll aufgezeigt werden, wo diese Ansätze Defizite aufweisen oder nicht anwendbar sind, z.B. weil sie von anderen Umgebungen oder Voraussetzungen ausgehen.

Je nach Art der Abschlussarbeit kann es auch sinnvoll sein, diesen Abschnitt in die Einleitung zu integrieren oder als eigenes Kapitel aufzuführen.

Beispiel, wie mit LaTeX zitiert werden kann: \cite{TB98,JSAC96,qosr}


State of the art compression
- techniques\\
- use cases\\

%%% Local Variables: 
%%% mode: latex
%%% TeX-master: "thesis"
%%% End: 
  % Grundlagen
%% analyse.tex
%% $Id: analyse.tex 61 2012-05-03 13:58:03Z bless $

\chapter{Analysis}
\label{ch:Analysis}
%% ==============================

\par{
The following chapter contains a detailed analysis of the problem and some fundamental requirements for the algorithm. 
}

%% ==============================
\section{Initial Findings}
%% ==============================
\label{ch:Analysis:sec:Initial Findings}

\par{
To have a comparison to some extent, the initial analysis was performed on the Galgary Corpus as well as the other well known compression methods. The results where very underwhelming as expeced, as shown in the table below.
}

\begin{center}
	\begin{tabular}[p]{r|r|r}
		%	\caption{Beschreibung der Tabelle}
		
		file &  original size & run length encoded size\\
		\hline
		0 &  00110101 & 0000110111\\
		1 & 000111 & 010\\
		2 & 0111 & 11\\
		3 & 1000 & 10\\
		4 & 1011 & 011\\
		... &  & \\
		20 & 0001000 & 00001101000\\
		... & & \\
		64+ & 11011 & 0000001111\\
		128+ & 10010 & 000011001000
		\label{tab:t4 run length Huffman codes}
	\end{tabular}
\end{center}



\par{
	Early findings\\
- no matrix representation but chunks of type byteArray\\
- static encoding difficulties\\
}

\par{
Binary RLE on bits of same significance\\
- similar to regular rle\\ 
}

\par{
Only combination kinda effective
}


\ldots

%% ==============================
\section{Improvements by Preprocessing}
%% ==============================
\label{ch:Analysis:sec:Improvements by Preprocessing}
- First improvements due to byte remapping\\
- burrows wheeler transformation\\
\ldots

%% ==============================
\section{Further Improvements}
%% ==============================
\label{ch:Analysis:sec:Further Improvements}
- combining different compression techniques\\
\ldots


%% ==============================
\section{Summary}
%% ==============================
\label{ch:Analyse:sec:Summary}

Am Ende sollten ggf. die wichtigsten Ergebnisse nochmal in \emph{einem}
kurzen Absatz zusammengefasst werden.

%%% Local Variables: 
%%% mode: latex
%%% TeX-master: "thesis"
%%% End: 
     % Analyse
%% entwurf.tex
%% $Id: entwurf.tex 61 2012-05-03 13:58:03Z bless $
%%

\chapter{Conceptual Design}
\label{ch:Conceptual Design}
%% ==============================
In diesem Kapitel erfolgt die ausführliche Beschreibung des eigenen
Lösungsansatzes. Dabei sollten Lösungsalternativen diskutiert und
Entwurfsentscheidungen dargelegt werden.

To archive the main objective a few ideas to improve run length encoding where found. The first real difference is the conception of reading the data in chunks and then encode it in parallel, meaning all the most significant bits in one chunk, then all second most significant bits and so on. The second change was to switch between the encoding of runs of characters \/ bytes to count runs of ones and zeros so basically the same mechanism but on bit level.

%% ==============================
\section{Parallel Byte Reading}
%% ==============================
\label{ch:Conceptual Design:sec:Parallel Byte Reading}
...
\subsection{First Ideas}
...
\subsection{New Perspective}
...
\subsection{Performance Improvements}
...
%% ==============================
\section{Preprocessing}
%% ==============================
\label{ch:Conceptual Design:sec:Preprocessing}
...

\subsection{Burrows Wheeler Transformation}
...
\subsection{Byte Mapping to reduce Input space}
...
\subsection{Dynamic Encoding}
...

\section{Alternative Compression for partial data}
\label{ch:Conceptual Design:sec:Alternative Encoding}
...
\subsection{Huffman Coding}
...
%% ==============================
\section{Summary}
%% ==============================
\label{ch:Conceptual Design:sec:Summary}

Am Ende sollten ggf. die wichtigsten Ergebnisse nochmal in \emph{einem}
kurzen Absatz zusammengefasst werden.

%%% Local Variables: 
%%% mode: latex
%%% TeX-master: "thesis"
%%% End: 
     % Entwurf
\chapter{Implementation}
\label{ch:Implementation}
%% ==============================
All algorithms described have been implemented using Kotlin, because it can be compiled for the Java Virtual Machine and for native execution, therefore it seemed like a good balance between a native implementation and higher language conciseness and fault tolerance. Also there were some libraries available for byte and bit operations on streams which proved to be quite useful, although there was almost no documentation available. This one and all other libraries are described in Section \ref{ch:Implementation:sec:Impl:subsec:libs}. The main focus is on the encoder and decoder classes but the other modules will be discussed as well. The project is realized as a maven project, to simplify dependency management. 

\section{Binary and Byte Wise RLE}
\label{ch:Implementation:bin and byte rle}
\par{
	The simple binary and byte wise RLE are rather trivial and implemented together in one encoder, the \textit{StringRunLengthEncoder}. The binary version is implemented with the mentioned BitStream from the IOStreams library. It allows working on a stream and reading and writing bit by bit and also reading the next $n$ bit as signed or unsigned number which comes in handy during decoding. In general, it is called with a variable $b$ bitsPerRun which sets the used bits to store one run. At first a maximum run length is determined by the maximum value $b$ can store as a binary string, $l(b) = n$ implies a maximum value of $2^{n-1}$. Then, the input is read consecutively in bits and the runs of equal bits are counted. We are always assuming the run starts with zero, if this is not the case a leading run of 0 is added, which means there are zero times 0 at the beginning. If a run exceeds the maximum run length, the maximum is written and again an artificial zero is added to the output stream to signal a length higher than the maximum. Each run can simply be written to the output stream with the desired amount of bits per run. During decoding, we assume the same $b$ and can then always read $n$ bits of the stream as unsigned value, know each run again and can therefore reconstruct the original data.
}
\par{
	Byte wise RLE is working with a similar idea of counting $n$ runs of equal information. This time it is applied on a byte level, reading byte after byte. If the next byte is the same as the current one, the counter is incremented, if not the run and the byte value are encoded as pair $(n, byte)$ to the output. The byte value itself still needs 8 bit of information but the run does not. Most raw untreated data does not contain long runs of consecutive identical byte values average run length is rather small. This implied storing a count of 1 or 2 in 8 bits of space which in turn explained the expansion in size seen in Table \ref{tab:t31 Byte-wise RLE on the Calgary Corpus}. Therefore, this version was also implemented with an arbitrary amount of bits to save per run, to minimize the overhead. If a run exceeds the maximum, which is again determined by the amount of bits stored per run, it is encoded twice, once with the maximum count and once with the remaining count. We do not need a zero run in this version because we also store the value itself, therefore we can count without a zero. This means for an example run of 4 times the value \textit{0xFF} and 4 bits per run saved, it will be encoded as the pair $(0011,0xFF)$ or $001111111111$ as consecutive binary stream.
}
\section{Vertical Binary RLE}
\label{ch:Implementation:vertical rle}
\par{
	Basically the ideas described in the Sections \ref{ch:Analysis:sec:Improvements by Preprocessing:subSec:vertReading} and \ref{ch:Conceptual Design:sec:Parallel Byte Reading} oppose only a small variance compared to regular binary RLE. It is realized with the use of BitStreams again. Its stream interface offers a position $p$, which corresponds to the byte value and an offset $o$, which is a bit value with significance $o$ of byte $p$, which allows reading all bits of the same significance in order. This was done for significance zero to seven to read all bits in a vertical manner as in the examples. Then, each run is again counted with the same method including a maximum run length defined by the amount of bits used to count a run and the runs are then written with the fixed amount of bits to the output stream. Afterwards, the amount of runs per bit position is written to the tail of the encoded file, without the information how many runs are expected it would be much more difficult to decide which run belongs where, but it is still possible. The average overhead of this additional information is around 34 byte, two for each count and a two byte stop symbol which is only needed seven times, no additional stop at the end of the file. Even though it was originally designed to work on chunks of bytes, in the end the transformations worked on the file or on the stream itself, which was significantly faster.
}
\par{
	During decoding, the expected amount of runs are parsed from the end of the file. Then the actual decoding happens, with the fixed $n$ bits per code for this bits significance. Knowing the exact amount of RLE numbers for each bit position makes it easy to decode, because the variable length of encoded numbers can be chosen accordingly while reading the stream once. Assuming a starting run on zero, all runs are written back to one file, each bit position sequentially, to then assemble the original data. This is done for each bit position in sequential order so we need to write 8 times to the output file. It might be worth trying out building the byte stream in memory and then write the output only once, which might be faster but also requires more internal data structures and holding the whole file in memory at once.
}
\section{Byte Remapping}
\label{ch:Implementation:bytemapping}
\par{
	To start of with the preprocessing, the byte remapping was implemented. The \emph{Analyzer} is responsible for generating an overall probability distribution over the values of bytes contained in the file. This serves as an input for the map generation, where every byte value is sorted accordingly to its occurrence and mapped to increasing byte values, so the most frequent byte to \textit{0x00}, the second most often to \textit{0x01} and so on. Afterwards, a temporary file is generated where each byte from the original file is mapped, which allows streaming during the encoding process. Decoding requires access to the original mapping, therefore it is persisted at the start of the encoded file. To do so, we only need to know the number of mapped values and then the original byte, not the mapped value since we know they are sorted acceding.
}
\section{Burrows Wheeler Transformation}
\label{ch:Implementation:bwt}
\par{
	As mentioned earlier, the Burrows-Wheeler-Transformation is implemented in 3 different versions, starting of with the naive vs.\ unsophisticated. The \emph{transformation.BurrowsWheelerTransformation} is implemented with the use of start and stop symbols (0x02 as STX and 0x03 as ETX) and with the creating and sorting of all cyclic rotations of the input string. This can be done for all text input files but not for binary data because files containing the start or stop symbols confused the algorithm and made the inverse transformation impossible. Additionally it is extremely slow due to its at least quadratic complexity, even when working on small chunks which messes up the overall transformation result. It is not further described as it was only used for some initial testing to see if and how much RLE benefits from this transformation. 
}
\par{
	The second implementation is realized by following the original algorithm description provided in greater detail in the paper by M. Burrows and D. J. Wheeler \cite{Burrows94} (algorithm C and D). The \emph{transformation.BurrowsWheelerTransformationModified} works on parts of the data so the transformation result is still strongly depending on the size of the chunks, but it could at least handle arbitrary input. Higher chunk sizes greatly increased the transformation because more equal characters are in the same chunk but also really slowed down the process. Due to the fact that both the mapping and the modified transformation work on an array of bytes and do not interfere with one-another, they can be performed in any order. The further on used advanced implementation of the bijective Burrows-Wheeler-Scott-Transformation is ported from Java from an external source or directly usable as dependency and therefore described in Section \ref{ch:Implementation:sec:Impl:subsec:libs}.
}
\section{Huffman Encoding}
\label{ch:Implementation:Huffman}
\par{
Following the pseudo-code provided by M. Liśkiewicz and H. Fernau in \cite{entropy-fernau} on page 21 the implementation was straight forward. Internally a small set of data structures are provided for assembling the Huffman tree, a \emph{HuffmanTree}, a \emph{HuffmanNode} and a \emph{HuffmanLeaf} class is implemented. The HuffmanTree is abstract and only holds the frequency of a tree, since every tree itself has its own frequency. It also implements a compareTo function, to draw comparisons between different trees. A HuffmanNode extends the HuffmanTree, consists of a left and right HuffmanTree and has their sum of frequencies as frequency. The leaf is itself also a tree and holds a value of type byte and a frequency which resembles its occurrence. To build the Huffman tree for a given set of bytes, the algorithm expects an array of integer $I$, assembling the occurrences $o$ of bytes $b \in [ 0,255 ]$ in the form of $I [ b ] = o$. Then a leaf is created for every entry of $I$ with a frequency of $o$ and collected into a PriorityQueue of HuffmanTrees. Then while there are still at least two trees left in the queue, the two lowest frequencies are removed from the queue, merged into a single tree and reinserted into the queue. After the creation of the tree, all paths are followed and every time a leaf is reached, the current path is added to a map in form of a StringBuffer $buf_b$. This buffer contains the path of left and right trees traversed into the original one, adding a zero for every left descent and a one for every right one. Finally the mapping of type byte to StringBuffer is returned. To apply this algorithm to the runs of the RLE encoding, the same is done except that instead byte values, every possible run length value is counted and collected into the same structure.
}
\par{
To reverse this Huffman coding it is required to have access to the performed mapping, otherwise decoding would be impossible. Therefore, the map itself is written to the beginning of the file, similar to the mapping from Section \ref{ch:Implementation:bytemapping}. This time though we need triplets of values, because the Huffman code can have a variable length, so the mapping is encoded to $(b,l(b),buf_b)$, with $b$ and $l(b)$ each assuming one byte. This also implies a maximum length for Huffman codes of 255. During decoding of the mapping, first the total number of mappings is parsed. Then, while there are still more mappings expected, we parse one byte which is the mapping value, then one byte which contains the length of the following mapping as unsigned byte value and then the Huffman code of the given length is parsed bit by bit and saved as a StringBuffer. This way it is possible to write and read the variable length codes continuously from the stream. 
}
\section{External Libraries}
\label{ch:Implementation:sec:Impl:subsec:libs}
\par{
Some external libraries are used throughout this project. Most of them provide a set of sealed functionality, like \emph{io.github.jupf.staticlog} which just facilitate the logging features and are not further interesting, therefore most of them are mentioned in Section \ref{ch:Implementation:sec:Impl:subsec:libs:others}, only extensively used ones are described in greater detail. 
}

\subsection{IOStreams for Kotlin}
\label{ch:Implementation:sec:Impl:subsec:libs:iostreams}
\par{
Alexander Kornilov created and released this library, which was found in the \href{https://discuss.kotlinlang.org/t/i-o-streams-for-kotlin/9802}{Kotlin forum}} and is currently hosted on \href{https://sourceforge.net/projects/kotlin-utils/}{Sourceforge}. It has a very light documentation \cite{IoStreamsKotlin} and is released with an Apache v2.0 license. It has to be mentioned that for the time being there is only a pre-release available, this version 0.33 is used throughout this project. Moving to this library rendered it possible to work entirely on streams of data as well as reading the next $n$ bits. This greatly improved the performance of the initial algorithm and reduced required data structures and memory. There was also some odd or unexpected behavior seen which might be changed in further versions, for example if the stream is currently at the first byte and we want to write to its bit with the highest significance. This represents a so called bit offset of 7, so we can write the position as 0:7. Writing a 1 or setting this bit to true advances the position of the stream to the next byte and offset zero, so 1:0 while writing a 0 keeps the stream at its current position so it is still at 0:7 and the next bit gets written on the same position. Basically the interface provides good functionality with the drawback of some inconsistencies.
}

\subsection{libDivSufSort}
\label{ch:Implementation:sec:Impl:subsec:libs:libDivSufSot}
\par{
The original code for the modified BWTS algorithm and the necessary sorting algorithms called \emph{DivSufSort} for efficient suffix array sorting was provided by Yuta Mori in the library \href{https://github.com/y-256/libdivsufsort}{libDivSufSort} \cite{LibDivSufSort} and is as already mentioned, closer described in the paper by Johannes Fischer and Florian Kurpicz \cite{DBLP:journals/corr/abs-1710-01896}. It  runs in $O(n \ log \ n)$ worst-case time using only $5n+O(1)$ bytes of memory space, where $n$ is the length of the input. The code is available under the MIT license at Github but written in C and thus, rather hard to use from Kotlin. Further research lead to a port to Java which uses the same structures and methods but is already close to the desired state, since Kotlin enables using Java classes by default.
}
\subsection{libDivSufSort in Java}
\label{ch:Implementation:sec:Impl:subsec:libs:libDivSufSort Java}
\par{
Porting the Java implementation provided by \href{https://github.com/flanglet/kanzi/releases}{Kanzi} \cite{kanzi}, a collection of state of the art compression methods, all available in Java, C++ and Go, was easy but also gratuitous, since there are releases available for the Java version which is still maintained at a high frequency. To use them we simply add the dependency to our maven project. This library basically provides one functionality, the implementation of a sophisticated bijective Burrows-Wheeler-Scott-Transformation in linear time.
}
\par{
To skim over its functionality, it provides a clean API to work with advanced compression algorithms in Java, most noticeable \emph{DivSufSort} and \emph{BWTS}. The BWTS class offers just the two directions of the transformation as methods, both working on arrays of bytes, which requires reading all the input into memory. This functionality is encapsulated by the class \emph{BWTSWrapper} to work on file level and generate a temporary file with the transformed contents to further work on a stream of that file. At first the whole input is spliced into Lyndon words. Then the suffix array is generated and sorted, using the fastest known suffix sorting algorithm DivSufSort. The result is also written to a temporary file, like the mapping which was quite useful for debugging, and enables streaming its contents during encoding. 
}

\subsection{Others}
\label{ch:Implementation:sec:Impl:subsec:libs:others}
\par{
Besides logging, some other basic functionality was added which was quite simple through the use of maven. \emph{org.junit.jupiter} provided the packages \emph{junit}, \emph{engine} and \emph{api} in version 5.6.0, which allows a simple test unit creation and execution. Additionally \emph{assert-j} in version 3.13.2 added extended assertion capabilities to the encoder as well as to the test cases. A modest approach of multithreading was enabled by use of Kotlin \emph{coroutines} in version 1.3.2, provided by Jetbrains. Just by convenience, Googles \emph{guava} was used in version 28.2 for new collection types and a comparator for an array of bytes.
}


%% ==============================
\section{Implementation Evaluation}
%% ==============================
\label{ch:Implementation:sec:Implementation Evaluation}
\par{
Obviously the assembled tool does not compete with the state of the art methods used today, neither in comparison of their compression results nor in terms of speed. There is most likely still a lot of unused potential to speed things up, for example by excessive multithreading. Also, decoding could be vastly speed up by writing each output byte only once instead of up to 8 times if the byte is of value 0xFF. Even compression results could probably be further improved but more on that in chapter \ref{ch:Discussion}. Nonetheless, the desired concept was proven and the results show a clear advantage over regular RLE achieved through preprocessing. 
}

%% ==============================
\section{Usage}
%% ==============================
\label{ch:Implementation:sec:usage}
\par{
To enable a convenient usage, the algorithm is obtainable as jar file but it can also be built from sources. It provides a simple command line interface which expects a desired action, either compressing \emph{-c} or decompressing \emph{-d} and a method, either vertical RLE \emph{-v}, binary \emph{-bin} or byte wise RLE \emph{-byte}. Binary and byte wise RLE can optionally be called with a parameter $N$ where $N$ is the amount of bits used to encode a single RLE number, vertical encoding can also run with an arbitrary amount of bits, but expects 8 comma separated numbers \emph{-v N,N,N,N,N,N,N,N}. As preprocessing options, mapping \emph{-map}, applying the sophisticated Burrows-Wheeler-Transformation \emph{-bwt}, and Huffman encoding \emph{-huf} are available via the parameters. Additional information can be acquired by launching the application with \emph{-h} for further help. To enable debug logging and get detailed insight into the compression and decompression steps, the parameter \emph{-D} has to be set. After all parameters have been set, a list of files is expected, started with the flag \emph{-f} and separated by space.
}

%%% Local Variables: 
%%% mode: latex
%%% TeX-master: "thesis"
%%% End: 
    % Implementierung
%% eval.tex
%% $Id: eval.tex 61 2012-05-03 13:58:03Z bless $

\chapter{Evaluation}
\label{ch:Evaluation}
%% ==============================
Moving on to the evaluation, we start by comparing the achieved results using different preprocessing options or combination of those. To start of it is clear that the best compression ratio was not accomplished by using a combination of different mapping techniques and a vertical interpretation of the input data but we now take a closer look at the discrepancy between each result. It is also clear that with the help of such methods, run length encoding can become a suitable compression algorithm for more than just pellet based images although it is clearly not as sophisticated as advanced state of the art compression methods mentioned in section \ref{ch:Principles of compression:sec:SOTA}.

%% ==============================
\section{Functional Evaluation}
%% ==============================
\label{ch:Evaluation:sec:Functional Evaluation}

\ldots
%% ==============================
\section{Benchmarks}
%% ==============================
\label{ch:Evaluation:sec:Benchmarks}
- Benchmark with the Calgary corpus\\
\url{http://www.data-compression.info/Corpora/} \\
\ldots

	\begin{table}[h]
	\begin{tabular}{r|r|r|r|r}
		method  &  size in bytes & compression & \textit{bps}& time\\
		\hline
		uncompressed & 3,145,718 & 100.0\% & 8.00 &\\
		compress & 1,250,382 & 40.4\% & 3.24 & 0.039s\\
		modified vertical RLE & 1237380 & 39.3\%& 3.14 & 16.45s\\
		gzip v1.10 & 1,021,720 & 32.4\% & 2.60 & 0.232s\\
		ZIP v3.0 & 1,019,783 & 32.4\% & 2.59 & 0.214s\\
		zstandard 1.4.2& 887,004 & 28.1\% & 2.25 & 0.951s\\
		bzip2 v1.0.8 & 832,443 & 26.4\% & 2.11 & 0.191s\\
		brotli & 826,638 & 26.3\%& 2.10 & 4.609s\\
		p7zip 16.02 (deflate) &  821,873 & 26.1\% & 2.08 & 0.431s \\
		p7zip 16.02 (PPMd) &  763,067& 24.2\% & 1.93 & 0.345s\\
		ZPAQ v7.15 & 659.700 & 20.9\% & 1.67 & 7.452s \\
		paq8hp* & - & - & - & - \\ 
		cmix v18 & 554,983 & 17.6\% & 1.41 & <3h		
	\end{tabular}
	\label{tab:t100benchmark}
	\caption{Benchmark on the Calgary Corpus}
\end{table}
%% ==============================
\section{Conclusion}
%% ==============================
\label{ch:Evaluation:sec:Conclusion}

Am Ende sollten ggf. die wichtigsten Ergebnisse nochmal in \emph{einem} kurzen Absatz zusammengefasst werden.

%%% Local Variables: 
%%% mode: latex
%%% TeX-master: "thesis"
%%% End: 
        % Evaluation
%% zusammenf.tex
%% $Id: zusammenf.tex 61 2012-05-03 13:58:03Z bless $
%%

\chapter{Discussion}
\label{ch:Discussion}
%% ==============================

\par{
	One interesting approach not performed in this scope, is the encoding of Huffman codes after a byte wise RLE. It was assumed to perform worse that the vertical encoding because there has to be one code for every combination of runs and values, thus very long average Huffman codes. Another idea was the substitution of Huffman encoding by another, more sophisticated method like Asymmetric Numeral Systems. This would most likely further improve compression results at the expense of slower computation.
}
%%% Local Variables: 
%%% mode: latex
%%% TeX-master: "thesis"
%%% End: 
   	  % Diskussion und Ausblick

%% ++++++++++++++++++++++++++++++++++++++++++
%% Anhang
%% ++++++++++++++++++++++++++++++++++++++++++

\appendix
%\include{anhang_a}
%\include{anhang_b}

%% ++++++++++++++++++++++++++++++++++++++++++
%% Literatur
%% ++++++++++++++++++++++++++++++++++++++++++
%  mit dem Befehl \nocite werden auch nicht 
%  zitierte Referenzen abgedruckt

\cleardoublepage
\phantomsection
\addcontentsline{toc}{chapter}{\bibname}
%%
%%\nocite{*} % nur angeben, wenn auch nicht im Text zitierte Quellen 
           % erscheinen sollen
%\bibliographystyle{itmabbrv} % mit abgekürzten Vornamen der Autoren
\bibliographystyle{ieeetr} % abbrvnat unsrtnat
% spezielle Zitierstile: Labels mit vier Buchstaben und Jahreszahl
%\bibliographystyle{itmalpha}  % ausgeschriebene Vornamen der Autoren

\bibliography{thesis}

%% ++++++++++++++++++++++++++++++++++++++++++
%% Index
%% ++++++++++++++++++++++++++++++++++++++++++
\ifnotdraft{
\cleardoublepage
\phantomsection
\printindex            % Index, Stichwortverzeichnis
}

 %
 % Die folgende Erklärung ist für Diplomarbeiten Pflicht
 % (siehe Prüfungsordnung), für Studienarbeiten nicht notwendig
 \thispagestyle{empty}
%\vspace*{35\baselineskip}
%\hbox to \textwidth{\hrulefill}
\par
\chapter*{Eidesstattliche Erklärung}

Hiermit erkläre ich, dass ich diese Bachelor-/Masterarbeit selbständig verfasst und keine anderen als die angegebenen Quellen und Hilfsmittel benutzt und die aus fremden Quellen direkt oder indirekt übernommenen Gedanken als solche kenntlich gemacht habe. Die Arbeit habe ich bisher keinem anderen Prüfungsamt in gleicher oder vergleichbarer Form vor-gelegt. Sie wurde bisher auch nicht veröffentlicht.

Trier, den xx. Monat 20xx

%%%%%%%%%%%%%%%%%%%%%%%%%%%%%%%%%%%%%%%%%%%%%%%%%%%%%%%%%%%%%%%%%%%%%%%%
%% Hinweis:
%%
%% Diese Erklärung wird von der Prüfungsordnung für Diplomarbeiten 
%% verlangt und ist zu unterschreiben. Für Studienarbeiten ist diese
%% Erklärung nicht zwingend notwendig, schadet aber auch nicht.
%%%%%%%%%%%%%%%%%%%%%%%%%%%%%%%%%%%%%%%%%%%%%%%%%%%%%%%%%%%%%%%%%%%%%%%%
\clearpage







 \blankpage % Leerseite auf Erklärungsrückseite
 
\end{document}
%% end of file

%% eval.tex
%% $Id: eval.tex 61 2012-05-03 13:58:03Z bless $

\chapter{Evaluation}
\label{ch:Evaluation}
%% ==============================
Moving on to the evaluation, we start by comparing the achieve results using different preprocessing options or combination of those. To start of it is clear that the best compression ratio 

%% ==============================
\section{Functional Evaluation}
%% ==============================
\label{ch:Evaluation:sec:Functional Evaluation}
- Mathematical Comparison\\
- Comparison of encoded file sizes\\
- comparing to regular REL and Huffman coding\\
\ldots

%% ==============================
\section{Benchmarks}
%% ==============================
\label{ch:Evaluation:sec:Benchmarks}
- Benchmark with the Galgary corpus\\
\url{http://www.data-compression.info/Corpora/} \\
\ldots

\subsection{Burrows-Wheeler-Transformation}

\par{
\begin{center}
	\begin{tabular}[p]{r|r|r}
		\label{tab:t6 run length eval bwt}
		
		bits per rle number & ratio in \% & bits per symbol in $\frac{bits}{symbol}$\\
		\hline
		3 & 95.4169856525027 & 7.633358852200216\\
		2 & 91.391 & 7.311309412577118 \\
	\end{tabular}
\end{center}
}


\subsection{Vertical encoding}
\par{


}

%% ==============================
\section{Conclusion}
%% ==============================
\label{ch:Evaluation:sec:Conclusion}

Am Ende sollten ggf. die wichtigsten Ergebnisse nochmal in \emph{einem} kurzen Absatz zusammengefasst werden.

%%% Local Variables: 
%%% mode: latex
%%% TeX-master: "thesis"
%%% End: 

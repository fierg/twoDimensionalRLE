%% analyse.tex
%% $Id: analyse.tex 61 2012-05-03 13:58:03Z bless $

\chapter{Analysis}
\label{ch:Analysis}
%% ==============================

\par{
The following chapter contains a detailed analysis of the problem and some fundamental requirements for the algorithm. 
}

%% ==============================
\section{Initial Findings}
%% ==============================
\label{ch:Analysis:sec:Initial Findings}

\par{
To have a comparison to some extent, the initial analysis was performed on the Galgary Corpus as well as the other well known compression methods. The results where very underwhelming as expeced, as shown in the table below.
}

\begin{center}
	\begin{tabular}[p]{r|r|r|r|r}
				\label{tab:t5 run length eval}
		
		bits per rle number & compression ratio in \% & bits per symbol in $\frac{bits}{symbol}$\\
		\hline
		8 & 165 & 13.2 \\
		7 & 154 & 12.38\\
		6 & 144 & 11.57 \\
		5 & 134 & 10.77\\
		4 & 125 & 10.00\\
		3 & 116 & 9.29\\
		2 & 109 & 8.74 \\
	\end{tabular}
\end{center}



\par{
	Early findings\\
- no matrix representation but chunks of type byteArray\\
- static encoding difficulties\\
}

\par{
Binary RLE on bits of same significance\\
- similar to regular rle\\ 
}

\par{
Only combination kinda effective
}


\ldots

%% ==============================
\section{Improvements by Preprocessing}
%% ==============================
\label{ch:Analysis:sec:Improvements by Preprocessing}
- First improvements due to byte remapping\\
- burrows wheeler transformation\\
\ldots

%% ==============================
\section{Further Improvements}
%% ==============================
\label{ch:Analysis:sec:Further Improvements}
- combining different compression techniques\\
\ldots


%% ==============================
\section{Summary}
%% ==============================
\label{ch:Analyse:sec:Summary}

Am Ende sollten ggf. die wichtigsten Ergebnisse nochmal in \emph{einem}
kurzen Absatz zusammengefasst werden.

%%% Local Variables: 
%%% mode: latex
%%% TeX-master: "thesis"
%%% End: 

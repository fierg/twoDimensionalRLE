\chapter*{Abstract}
%% ==============================
The paper reviews a combination of preprocessing methods, such as a Burrows-Wheeler-Transformation, applied to the coding scheme Run Length Encoding and analyzes their effects to the compression ratio. By comparing and applying different methods and their combinations, the encoding scheme is able to work on arbitrary data. It is shown that the implemented preprocessing steps enable Run Length Encoding to achieve a reasonable compression on two corpora at an acceptable speed. Furthermore, the results are discussed in detail and the implementation is depicted.

\paragraph{Übersetzung}
Diese Arbeit thematisiert eine Kombination von Vorverarbeitungsverfahren, wie z. B. einer Burrows-Wheeler-Transformation, die auf das Codierungsschema Run Length Encoding angewendet wird und analysiert deren Auswirkungen auf das Kompressionsverhältnis. Verschiedene Methoden und deren Kombinationen werde evaluiert und mit dem Codierungsschema auf beliebige Daten erfolgreich angewendet. Es wird gezeigt, dass die implementierten Vorverarbeitungsschritte es der Lauflängencodierung ermöglichen, eine angemessene Komprimierung für zwei Korpora mit einer akzeptablen Geschwindigkeit zu erzielen. Darüber hinaus werden die Ergebnisse detailliert besprochen und die Implementierung dargestellt.
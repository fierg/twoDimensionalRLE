%% entwurf.tex
%% $Id: entwurf.tex 61 2012-05-03 13:58:03Z bless $
%%

\chapter{Conceptual Design}
\label{ch:Conceptual Design}
%% ==============================
In diesem Kapitel erfolgt die ausführliche Beschreibung des eigenen
Lösungsansatzes. Dabei sollten Lösungsalternativen diskutiert und
Entwurfsentscheidungen dargelegt werden.

To archive the main objective a few ideas to improve run length encoding where found. The first real difference is the conception of reading the data in chunks and then encode it in parallel, meaning all the most significant bits in one chunk, then all second most significant bits and so on. The second change was to switch between the encoding of runs of characters \/ bytes to count runs of ones and zeros so basically the same mechanism but on bit level.

%% ==============================
\section{Parallel Byte Reading}
%% ==============================
\label{ch:Conceptual Design:sec:Parallel Byte Reading}
...
\subsection{First Ideas}
...
\subsection{New Perspective}
...
\subsection{Performance Improvements}
...
%% ==============================
\section{Preprocessing}
%% ==============================
\label{ch:Conceptual Design:sec:Preprocessing}
...

\subsection{Burrows Wheeler Transformation}
...
\subsection{Byte Mapping to reduce Input space}
...
\subsection{Dynamic Encoding}
...

\section{Alternative Compression for partial data}
\label{ch:Conceptual Design:sec:Alternative Encoding}
...
\subsection{Huffman Coding}
...
%% ==============================
\section{Summary}
%% ==============================
\label{ch:Conceptual Design:sec:Summary}

Am Ende sollten ggf. die wichtigsten Ergebnisse nochmal in \emph{einem}
kurzen Absatz zusammengefasst werden.

%%% Local Variables: 
%%% mode: latex
%%% TeX-master: "thesis"
%%% End: 

%% analyse.tex
%% $Id: analyse.tex 61 2012-05-03 13:58:03Z bless $

\chapter{Analyse}
\label{ch:Analyse}
%% ==============================
In diesem Kapitel sollen zunächst das zu lösende Problem
sowie die Anforderungen und die Randbedingungen 
einer Lösung\index{Lösung} beschrieben werden (eine präzisierte Aufgabenstellung\index{Aufgabenstellung}).


%% ==============================
\section{Anforderungen}
%% ==============================
\label{ch:Analyse:sec:Anforderungen}
Anforderungen und Randbedingungen\index{Randbedingungen} \ldots

%% ==============================
\section{Weiterer Abschnitt}
%% ==============================
\label{ch:Analyse:sec:Abschnitt}

Lorem ipsum dolor sit amet, consetetur sadipscing elitr, sed diam nonumy eirmod tempor invidunt ut labore et dolore magna aliquyam erat, sed diam voluptua. At vero eos et accusam et justo duo dolores et ea rebum. Stet clita kasd gubergren, no sea takimata sanctus est Lorem ipsum dolor sit amet.

\begin{figure}[htb]
\centering
  	{\includegraphics[width=.3\textwidth]{Logo-Uni-Trier.jpg}}
	\caption{Logo der Universität Trier.\label{fig:grafik1}}
\centering
\end{figure}

Lorem ipsum dolor sit amet, consetetur sadipscing elitr, sed diam nonumy eirmod tempor invidunt ut labore et dolore magna aliquyam erat, sed diam voluptua. 

\begin{table}[htb]
\caption{Tabelle mit Werten.\label{tab:liste}}
\vspace*{1em}
\centering

\bgroup
\def\arraystretch{1.3}%  1 is the default, change whatever you need

\begin{tabular}[c]{l|l|c}
	
	\multicolumn{1}{c|}{\textbf{A}} & 
	\multicolumn{1}{c|}{\textbf{B}} & 
	\multicolumn{1}{c}{\textbf{C}} \\ 
	
	\hline

	Test 1& Slow& 279 \\ 
	&Fast & 499 \\ 
	&Very fast& 719 \\ 
	
\end{tabular}

\egroup

\end{table}

Duis autem vel eum iriure dolor in hendrerit in vulputate velit esse molestie consequat, vel illum dolore eu feugiat nulla facilisis at vero eros et accumsan et iusto odio dignissim qui blandit praesent luptatum zzril delenit augue duis dolore te feugait nulla facilisi. 

Duis autem vel eum iriure dolor in hendrerit in vulputate velit esse molestie consequat, vel illum dolore eu feugiat nulla facilisis. 

%% ==============================
\section{Zusammenfassung}
%% ==============================
\label{ch:Analyse:sec:zusammenfassung}

Am Ende sollten ggf. die wichtigsten Ergebnisse nochmal in \emph{einem}
kurzen Absatz zusammengefasst werden.

%%% Local Variables: 
%%% mode: latex
%%% TeX-master: "thesis"
%%% End: 
